%!TeX root = main.tex
\documentclass{article}
\usepackage{graphicx} % Required for inserting images
\usepackage{listings}
\usepackage{ctex}
\usepackage{fontspec}
\usepackage{geometry}
\usepackage{fancyhdr}
\usepackage{lastpage}

\pagestyle{fancy}
\fancyhf{}
\fancyfoot[C]{\thepage\ /\ \pageref{LastPage}}
\renewcommand{\headrulewidth}{0pt}
\renewcommand{\textasteriskcentered}{*} % fuck !

\geometry{a4paper, top=3cm, bottom=3.5cm, left=3cm, right=3cm}

\newfontfamily\cp[
    Path=font/, 
    Extension=.ttf,
    BoldFont={* Bold}, 
    ItalicFont={* Italic},
    UprightFont={*}
]{Courier Prime}

\newfontfamily\jbmn[
    Path=font/, 
    Extension=.ttf,
    BoldFont={*-Bold}, 
    ItalicFont={*-Italic},
    UprightFont={*-Regular}
]{JetBrainsMonoNL}

\newcommand{\codefont}{\cp}

\lstset{
    basicstyle          =   \codefont,          % 基本代码风格
    keywordstyle        =   \bfseries,          % 关键字风格
    commentstyle        =   \codefont\itshape,  % 注释的风格,斜体
    stringstyle         =   \codefont,  % 字符串风格
    columns             =   fixed,
    numbers             =   left,   % 行号的位置在左边
    showspaces          =   false,  % 是否显示空格,显示了有点乱,所以不现实了
    numberstyle         =   \zihao{-5}\ttfamily,    % 行号的样式,小五号,tt等宽字体
    showstringspaces    =   false,
    captionpos          =   t,      % 这段代码的名字所呈现的位置,t指的是top上面
    frame               =   l,   % 显示边框
    breaklines          =   true,
}

\lstdefinelanguage{vim}{
    morecomment         =   [l]{\"},
}

\title{\textbf{Template}}
\author{Billy Wang}
\date{\today}

\begin{document}

\maketitle
\thispagestyle{fancy}
\tableofcontents
\newpage

\section{写在前面}

\subsection{基础模版}

\lstinputlisting[language=C++]{./code/1/1.1-demo.cpp}

\subsection{vimrc}

\lstinputlisting[language=vim]{./code/1/1.2-vimrc}

\section{数据结构}

\subsection{zkw 线段树}

单点修 区间查

\lstinputlisting[language=C++]{./code/2/2.1-zkw.cpp}

\subsection{珂朵莉树}

\lstinputlisting[language=C++]{./code/2/2.2-odt.cpp}

\subsection{FHQ-Treap}

以模版文艺平衡树为例

\lstinputlisting[language=C++]{./code/2/2.3-fhq.cpp}

\section{数学}

\subsection{快速幂}

\lstinputlisting[language=C++]{./code/3/3.1-qpow.cpp}

\subsection{高斯消元}

\lstinputlisting[language=C++]{./code/3/3.2-elimination.cpp}

\section{图论}

\subsection{倍增}

\lstinputlisting[language=C++]{./code/4/4.1-lca.cpp}

\subsection{网络流}

不是我写的,但是看着还好

其中 \lstinline{ll} 是我改的,不敢保证有没有漏改,但是过了洛谷模版题

\subsubsection{最大流}

\lstinputlisting[language=C++]{./code/4/4.2.1-dinic.cpp}

\subsubsection{费用流}

\lstinputlisting[language=C++]{./code/4/4.2.2-mcmf.cpp}

\subsection{二分图最大匹配}

ps. 建单向图(即只有左部指向右部的边)

\lstinputlisting[language=C++]{./code/4/4.3-maxmatch.cpp}

\subsection{Tarjan 强连通分量缩点}

\lstinputlisting[language=C++]{./code/4/4.4-tarjanscc.cpp}

\end{document}
