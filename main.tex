%!TeX root = main.tex
\documentclass{article}
\usepackage{graphicx} % Required for inserting images
\usepackage{listings}
\usepackage{ctex}
\usepackage{fontspec}
\usepackage{geometry}
\usepackage{fancyhdr}
\usepackage{lastpage}
\usepackage[colorlinks,linkcolor=black]{hyperref}

\pagestyle{fancy}
\fancyhf{}
\fancyfoot[C]{\thepage\ /\ \pageref{LastPage}}
\renewcommand{\headrulewidth}{0pt}
\renewcommand{\textasteriskcentered}{*} % fuck !

\geometry{a4paper, top=3cm, bottom=3.5cm, left=3cm, right=3cm}

\newfontfamily\cp[
    Path=font/, 
    Extension=.ttf,
    BoldFont={* Bold}, 
    ItalicFont={* Italic},
    UprightFont={*}
]{Courier Prime}

\newfontfamily\jbmn[
    Path=font/, 
    Extension=.ttf,
    BoldFont={*-Bold}, 
    ItalicFont={*-Italic},
    UprightFont={*-Regular}
]{JetBrainsMonoNL}

\newcommand{\codefont}{\cp}

\lstset{
    basicstyle          =   \codefont,          % 基本代码风格
    keywordstyle        =   \bfseries,          % 关键字风格
    commentstyle        =   \codefont\itshape,  % 注释的风格,斜体
    stringstyle         =   \codefont,  % 字符串风格
    columns             =   fixed,
    numbers             =   left,   % 行号的位置在左边
    showspaces          =   false,  % 是否显示空格,显示了有点乱,所以不现实了
    numberstyle         =   \zihao{-5}\ttfamily,    % 行号的样式,小五号,tt等宽字体
    showstringspaces    =   false,
    captionpos          =   t,      % 这段代码的名字所呈现的位置,t指的是top上面
    frame               =   l,   % 显示边框
    breaklines          =   true,
}

\lstdefinelanguage{vim}{
    morecomment         =   [l]{\"},
}

\title{\textbf{Template}}
\author{Billy Wang}
\date{\today}

\begin{document}

\maketitle
\thispagestyle{fancy}
\tableofcontents
\newpage

\section{写在前面}

\subsection{基础模版}

\lstinputlisting[language=C++]{./code/1/1.1-demo.cpp}

\subsection{vimrc}

\lstinputlisting[language=vim]{./code/1/1.2-vimrc}

\section{数据结构}

\subsection{zkw 线段树}

单点修 区间查

\lstinputlisting[language=C++]{./code/2/2.1-zkw.cpp}

\subsection{珂朵莉树}

\lstinputlisting[language=C++]{./code/2/2.2-odt.cpp}

\subsection{FHQ-Treap}

以模版文艺平衡树为例

\lstinputlisting[language=C++]{./code/2/2.3-fhq.cpp}

\subsection{并查集}

\lstinputlisting[language=C++]{./code/2/2.4-dsu.cpp}

\subsection{ST 表}

\lstinputlisting[language=C++]{./code/2/2.5-st.cpp}

\subsection{树状数组}

\lstinputlisting[language=C++]{./code/2/2.6-fwt.cpp}

\subsection{线段树}

\section{数学}

\subsection{快速幂}

\lstinputlisting[language=C++]{./code/3/3.1-qpow.cpp}

\subsection{高斯消元}

\lstinputlisting[language=C++]{./code/3/3.2-elimination.cpp}

\subsection{筛法}

\subsubsection{埃式筛}

\lstinputlisting[language=C++]{./code/3/3.3.1-erato.cpp}

\subsubsection{线性筛}

\lstinputlisting[language=C++]{./code/3/3.3.2-euler.cpp}

\subsection{类欧几里得}

\subsection{递推组合数}

\lstinputlisting[language=C++]{./code/3/3.5-calcc.cpp}

\subsection{矩阵快速幂} % 模数为参数

\lstinputlisting[language=C++]{./code/3/3.6-matqpow.cpp}

\subsection{扩展欧几里得} % 模数为参数

\lstinputlisting[language=C++]{./code/3/3.7-exgcd.cpp}

\section{图论}

\subsection{倍增}

\lstinputlisting[language=C++]{./code/4/4.1-lca.cpp}

\subsection{网络流}

不是我写的,但是看着还好

其中 \lstinline{ll} 是我改的,不敢保证有没有漏改,但是过了洛谷模版题

\subsubsection{最大流}

\lstinputlisting[language=C++]{./code/4/4.2.1-dinic.cpp}

\subsubsection{费用流}

\lstinputlisting[language=C++]{./code/4/4.2.2-mcmf.cpp}

\subsection{二分图最大匹配}

ps. 建单向图(即只有左部指向右部的边)

\lstinputlisting[language=C++]{./code/4/4.3-maxmatch.cpp}

\subsection{Tarjan 强连通分量缩点}

\lstinputlisting[language=C++]{./code/4/4.4-tarjanscc.cpp}

\subsection{树直径}

\lstinputlisting[language=C++]{./code/4/4.5-diameter.cpp}

\subsection{树重心}

\lstinputlisting[language=C++]{./code/4/4.6-centroid.cpp}

\subsection{树链剖分}

\subsection{最短路}

\subsubsection{Floyd(最小环)}

\lstinputlisting[language=C++]{./code/4/4.8.1-floyd.cpp}

\subsubsection{Spfa(判负环)}

\lstinputlisting[language=C++]{./code/4/4.8.2-spfa.cpp}

\subsubsection{Dijkstra}

\lstinputlisting[language=C++]{./code/4/4.8.3-dijkstra.cpp}

\subsection{拓扑排序}

\lstinputlisting[language=C++]{./code/4/4.9-topo.cpp}

\subsection{最小生成树}

Kruskal 算法

\lstinputlisting[language=C++]{./code/4/4.10-kruskal.cpp}

\subsection{欧拉路径/回路}

\lstinputlisting[language=C++]{./code/4/4.11-eulerpath.cpp}

\section{字符串}

\subsection{KMP}

\lstinputlisting[language=C++]{./code/5/5.1-kmp.cpp}

\subsection{Trie 树}

\section{STL}

\subsection{算法库}

\textbf{不修改序列的操作}

\noindent \textbf{批量操作} 

\textit{在标头 \lstinline{<algorithm>} 定义}

\noindent \lstinline{for_each}

应⽤⼀元函数对象到范围中元素 (函数模板)

\noindent \lstinline{ranges::for_each} (C++20) 

应⽤⼀元函数对象到范围中元素 (算法函数对象)

\noindent \lstinline{for_each_n} (C++17) 

应⽤函数对象到序列的前 N 个元素 (函数模板)

\noindent \lstinline{ranges::for_each_n} (C++20) 

应⽤函数对象到序列的前 N 个元素 (算法函数对象)

\noindent \textbf{搜索操作} 

\textit{在标头 \lstinline{<algorithm>} 定义}

\noindent \lstinline{all_of} (C++11)

\noindent \lstinline{any_of} (C++11)

\noindent \lstinline{none_of} (C++11)

检查谓词是否对范围中所有、任⼀或⽆元素为 \lstinline{true} (函数模板)

\noindent \lstinline{ranges::all_of} (C++20)

\noindent \lstinline{ranges::any_of} (C++20)

\noindent \lstinline{ranges::none_of} (C++20)

检查谓词是否对范围中所有、任⼀或⽆元素为 \lstinline{true} (算法函数对象)

\noindent \lstinline{ranges::contains} (C++23)

\noindent \lstinline{ranges::contains_subrange} (C++23)

检查范围是否包含给定元素或⼦范围 (算法函数对象)

\noindent \lstinline{find}

\noindent \lstinline{find_if}

\noindent \lstinline{find_if_not} (C++11)

查找⾸个满⾜特定条件的元素 (函数模板)

\noindent \lstinline{ranges::find} (C++20)

\noindent \lstinline{ranges::find_if} (C++20)

\noindent \lstinline{ranges::find_if_not} (C++20)

查找⾸个满⾜特定条件的元素 (算法函数对象)

\noindent \lstinline{ranges::find_last} (C++23)

\noindent \lstinline{ranges::find_last_if} (C++23)

\noindent \lstinline{ranges::find_last_if_not} (C++23)

查找最后⼀个满⾜特定条件的元素 (算法函数对象)

\noindent \lstinline{find_end}

查找元素序列在特定范围中最后⼀次出现 (函数模板)

\noindent \lstinline{ranges::find_end} (C++20)

查找元素序列在特定范围中最后⼀次出现 (算法函数对象)

\noindent \lstinline{find_first_of} 

搜索⼀组元素中任⼀元素 (函数模板)

\noindent \lstinline{ranges::find_first_of} (C++20)

搜索⼀组元素中任⼀元素 (算法函数对象)

\noindent \lstinline{adjacent_find}

查找⾸对相同(或满⾜给定谓词)的相邻元素 (函数模板)

\noindent \lstinline{ranges::adjacent_find} (C++20) 

查找⾸对相同(或满⾜给定谓词)的相邻元素 (算法函数对象)

\noindent \lstinline{count}

\noindent \lstinline{count_if}

返回满⾜特定条件的元素数⽬ (函数模板)

\noindent \lstinline{ranges::count} (C++20)

\noindent \lstinline{ranges::count_if} (C++20)

返回满⾜特定条件的元素数⽬ (算法函数对象)


\noindent \lstinline{mismatch}

查找两个范围的⾸个不同之处 (函数模板)

\noindent \lstinline{ranges::mismatch} (C++20)

查找两个范围的⾸个不同之处 (算法函数对象)

\noindent \lstinline{equal}

判断两组元素是否相同 (函数模板)

\noindent \lstinline{ranges::equal} (C++20)

判断两组元素是否相同 (算法函数对象)

\noindent \lstinline{search}

搜索元素范围的⾸次出现 (函数模板)

\noindent \lstinline{ranges::search} (C++20)

搜索元素范围的⾸次出现 (算法函数对象)

\noindent \lstinline{search_n}

搜索元素在范围中⾸次连续若⼲次出现 (函数模板)

\noindent \lstinline{ranges::search_n} (C++20) 

搜索元素在范围中⾸次连续若⼲次出现 (算法函数对象)

\noindent \lstinline{ranges::starts_with} (C++23)

检查⼀个范围是否始于另⼀范围 (算法函数对象)

\noindent \lstinline{ranges::ends_with} (C++23)

检查⼀个范围是否终于另⼀范围 (算法函数对象)

\end{document}
